\documentclass[] {article}

\usepackage[T1]{fontenc}
\usepackage[utf8]{inputenc}
\usepackage{polski}
\usepackage{url}
\usepackage{verbatim}
\usepackage{geometry}

\title{Specyfikacja implementacyjna\\Projekt zespołowy}
\author{Damian Wróblewski\\Bartłomiej Łukasik\\Karol Kociołek}
\begin{document}

\maketitle

\section{Informacje ogólne}
Program napisany będzie w języku Java\\
Po uruchomieniu wyświetla GUI do obsługi programu

\section{Struktura projektu}
Klasy:
\begin{itemize}
	\item Main - klasa sterująca programem
	\item GUI - klasa odpowiająca za graficzny interfejs użytkownika
	\item Utils - klasa obsługująca wejścia i wyjścia i dodająca potrzebne metody
	\item AnimationPanel - klasa przedstawiająca animacje
	\item SettingsPanel - klasa odpowiadająca za intefejs ustawień
	\item Algorithms - klasa zawierająca algorytmy
\end{itemize}

\section{Sposób działania algorytmu}
\begin {enumerate}
	\item \textbf{Wczytanie danych z pliku i zliczenie sumy maksymalnej liczby szczepionek, które może kupić każda z aptek}
	\item \textbf{Sortowanie tablicy kontraktów według ceny jednej szczepionki} \\
			Wykorzystanie metody sortowania szybkiego
	\item  \textbf{Uzupełnienie minimalnej liczby szczepionek do kupienia dla każdej umowy }\\
			Jeśli ( maksymalna suma szczepionek do kupienia przez daną aptekę - zapotrzebowanie tej aptek < maksymalna liczba szczpionek do kupienia przez daną aptekę według jednej umowy ), to (liczba szczepionek do kupienia według danej umowy = maksymalna  liczba szczepionek do kupienia - ( maksymalna suma szczepionek do kupienia przez daną aptekę - zapotrzebowanie tej aptek ))
	\item \textbf{Przypisywanie liczby szczepionek do kupienia idąc zgodnie z kolejnościa posortowanej tablicy umów} \\
			Liczba szczepionek do kupienia = min( maksymalna liczba szczepionek do kupienia według umowy, liczba dostępnych do kupienia szczepionek od danej fabryki, liczba brakujących szczepionek dla danej apteki)

\end{enumerate}

\section{Testowanie}
Testy zostaną wykonane z pomocą narzędzia JUnit.\\
GUI testowane ręcznie

\end{document}